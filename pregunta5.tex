
\question[20]
Suponga que un electr\'{o}n est\'{a} en un estado descrito por la funci\'{o}n de onda

\begin{equation*}
    \psi=\dfrac{1}{\sqrt{4 \pi}} \left( e^{i \phi} sin \theta + cos \theta \right) g(r),
\end{equation*}
donde  
$$
\int_0 ^{\infty} |g(r)|^2 r^2 dr=1,
$$
y $\phi$, $\theta$ son los \'{a}ngulos azimutal y polar respectivamente.

\begin{parts}
  \part[10] ?`Cu\'{a}les son los posibles resultados de una medida de la componente $z$, i.e., $L_z$, del momento angular del electr\'{o}n en este estado?
  \part[5] ?`Cu\'{a}l es la probabilidad de obtener cada uno de los posibles valores en el anterior literal?
	\part[5] ?`Cu\'{a}l es el valor esperado de $L_z$?
\end{parts}

\begin{solution}
  \begin{parts}
	\item Como se puede ver en la tabla 5.2 de Zetilli,
	\begin{itemize*}
	    \item[] $Y_{10}(\phi,\theta)=\sqrt{\dfrac{3}{4 pi}} cos \theta$
	    \item[] $Y_{1 \pm 1}(\phi,\theta)=\mp \sqrt{\dfrac{3}{8 \pi}}e^{\pm i \phi} sin \theta$
	\end{itemize*}
	Entonces, la funci\'{o}n de onda puede escribirse como:
 \begin{equation*}
     \psi=\sqrt{\dfrac{1}{3}} \left( -\sqrt{2} Y_{11} + Y_{10}\right)g(r)
 \end{equation*}
	Veamos si $\psi$ esta normalizada:
\begin{equation*}
\begin{aligned}
 \int \psi^{*} \psi dV&=\dfrac{1}{3}\int \left( -\sqrt{2}Y_{11}^{*} + Y_{10}^{*}\right)g(r)^{*} \left( -\sqrt{2} Y_{11}+Y_{10}\right)g(r) r^2 sin \theta d \theta d\phi dr \\
 &= \dfrac{1}{3}\int_0 ^{\infty} |g(r)|^2 r^2 dr \int \left( 2|Y_{11}|^2 + |Y_{10}|^2  \right)sin \theta d\theta d\phi\\
 &=\dfrac{1}{3} \left( 1\right) \left(2+1\right)\\
 &=1
 \end{aligned}\label{Norm}
\end{equation*}
Se han utilizado los resultados 5.164 y 5.165 de Zetilli.
\\
Por lo tanto, los posibles valores de $L_z$ son $\hbar$ y $0$.

\item Como la funci\'{o}n de onda  esta normalizada, la probabilidad de obtener $L_z=\hbar$ es $\left(\sqrt{\dfrac{2}{3}}\right)^2=\dfrac{2}{3}$. Mientras que la probabilidad de tener $L_z=0$ es $\left(\dfrac{1}{\sqrt{3}}\right)^2=\dfrac{1}{3}$.

\item 
\begin{equation*}
\begin{aligned}
 \langle L_z \rangle &=\langle \psi| L_z |\psi \rangle \\
 &= \int \psi^{*} L_z \psi r^2 sin \theta d \theta d\phi \dr\\
 &=\dfrac{1}{3}\int \left( -\sqrt{2}Y_{11}^{*} + Y_{10}^{*}\right)g(r)^{*} \hat{L_z} \left( -\sqrt{2} Y_{11}+Y_{10}\right)g(r) r^2 sin \theta d \theta d\phi dr\\
\end{aligned}\label{Norm}
\end{equation*}

Puesto que $\hat{L_z}Y_{11}=\hbar Y_{11}$\\
y que $\hat{L_z}Y_{10}=0$ la integral se transforma en:\\
\begin{equation*}
    \langle L_z \rangle = \dfrac{2}{3}\hbar \int_{0}^{2\pi} d\phi \int_{0}^{\pi} d\theta sin \theta |Y_{11}|^2
\end{equation*}
	Tomando el resultado 5.165 de Zetilli se tiene $\langle L_z \rangle=\dfrac{2}{3}\hbar$
\end{parts}
\end{solution}

\end{questions}






 
