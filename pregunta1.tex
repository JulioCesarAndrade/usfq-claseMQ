\question [1] (20 puntos) Explique brevemente qué tiene que ver el momento magnético de las partículas con el momento angular, y cómo esto es importante en mecánica cuántica.

\begin{solution}

Clásicamente el momento magnetico aparece (o se define) a partir de una corriente en un lazo. Esto puede ser asociado al movimiento orbital de una partícula cargada o a su giro intrínseco. Por lo tanto, el momento magnético es proporcional ya sea al momento algular o al espín.
Esto es impotante en mecánica cuántica porque partículas con espín, por ejemplo, pueden acoplarse (o sentir) la influencia de campos magnéticos. 


\end{solution}

vspace{1.0cm}