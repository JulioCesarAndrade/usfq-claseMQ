%
% Plantilla para deberes o exámenes
%

%\documentclass[11pt]{exam}
\documentclass[answers,11pt]{exam}
% ***********************************************************
% ******************* PHYSICS HEADER ************************
% ***********************************************************
% Version 2
\usepackage{amsmath} % AMS Math Package
\usepackage{braket}  %for notation
\usepackage{amsthm} % Theorem Formatting
\usepackage{amssymb}	% Math symbols such as \mathbb
\usepackage{graphicx} % Allows for eps images
\usepackage{multicol} % Allows for multiple columns
 % Sets margins and page size
\pagestyle{empty} % Removes page numbers
\makeatletter % Need for anything that contains an @ command 
\renewcommand{\maketitle} % Redefine maketitle to conserve space
{ \begingroup \vskip 10pt \begin{center} \large {\bf \@title}
	\vskip 10pt \large \@author \hskip 20pt \@date \end{center}
  \vskip 10pt \endgroup \setcounter{footnote}{0} }
\makeatother % End of region containing @ commands
\renewcommand{\labelenumi}{(\alph{enumi})} % Use letters for enumerate
% \DeclareMathOperator{\Sample}{Sample}
\let\vaccent=\v % rename builtin command \v{} to \vaccent{}
\renewcommand{\v}[1]{\ensuremath{\mathbf{#1}}} % for vectors
\newcommand{\gv}[1]{\ensuremath{\mbox{\boldmath$ #1 $}}} 
% for vectors of Greek letters
\newcommand{\uv}[1]{\ensuremath{\mathbf{\hat{#1}}}} % for unit vector
\newcommand{\abs}[1]{\left| #1 \right|} % for absolute value
\newcommand{\avg}[1]{\left< #1 \right>} % for average
%\let\underdot=\d % rename builtin command \d{} to \underdot{}
%\renewcommand{\d}[2]{\frac{d #1}{d #2}} % for derivatives
\newcommand{\dd}[2]{\frac{d^2 #1}{d #2^2}} % for double derivatives
\newcommand{\pd}[2]{\frac{\partial #1}{\partial #2}} 
% for partial derivatives
\newcommand{\pdd}[2]{\frac{\partial^2 #1}{\partial #2^2}} 
% for double partial derivatives
\newcommand{\pdc}[3]{\left( \frac{\partial #1}{\partial #2}
 \right)_{#3}} % for thermodynamic partial derivatives
%\newcommand{\ket}[1]{\left| #1 \right>} % for Dirac bras
%\newcommand{\bra}[1]{\left< #1 \right|} % for Dirac kets
%\newcommand{\braket}[2]{\left< #1 \vphantom{#2} \right|\left. #2 \vphantom{#1} \right>} % for Dirac brackets
\newcommand{\matrixel}[3]{\left< #1 \vphantom{#2#3} \right|
 #2 \left| #3 \vphantom{#1#2} \right>} % for Dirac matrix elements
\newcommand{\grad}[1]{\gv{\nabla} #1} % for gradient
\let\divsymb=\div % rename builtin command \div to \divsymb
\renewcommand{\div}[1]{\gv{\nabla} \cdot #1} % for divergence
\newcommand{\curl}[1]{\gv{\nabla} \times #1} % for curl
\let\baraccent=\= % rename builtin command \= to \baraccent
\renewcommand{\=}[1]{\stackrel{#1}{=}} % for putting numbers above =
\newtheorem{prop}{Proposition}
\newtheorem{thm}{Theorem}[section]
\newtheorem{lem}[thm]{Lemma}
\theoremstyle{definition}
\newtheorem{dfn}{Definition}
\theoremstyle{remark}
\newtheorem*{rmk}{Remark}

\usepackage{amsmath, amssymb, graphics, setspace}

\newcommand{\mathsym}[1]{{}}
\newcommand{\unicode}[1]{{}}

\newcounter{mathematicapage}

% ***********************************************************
% ********************** END HEADER *************************
% ***********************************************************

%nice geometry
\usepackage[a4paper,left=3.5cm, right=3.5cm, top=3cm, bottom=2.3cm]{geometry}

%avoid numbering of pages
\pagestyle{empty}

%figures and html
\usepackage{epsfig}
%\usepackage{html}
\usepackage[final]{pdfpages}
%wrap text around figures
\usepackage{wrapfig}
\usepackage{float}


%stuff needed for spanish
\usepackage[spanish]{babel}
\selectlanguage{spanish}
\decimalpoint 
%\usepackage[latin1]{inputenc}
\usepackage[utf8]{inputenc}
%espacio vertical
%change Solution to Solución
\renewcommand{\solutiontitle}{\noindent\textbf{Solución:}\enspace}

\begin{document}

\begin{center}
{\bf UNIVERSIDAD SAN FRANCISCO DE QUITO\\
MAESTRÍA EN FÍSICA
MECÁNICA CUÁNTICA \\ 
Examen Parcial \#2 \\ 
Fecha: Abril 13, 2020 \\
Puntaje Máximo: 100/100 pts\\
Tiempo máximo: 110 minutos \\ 
Notas y texto cerrados.\\
}
\end{center}
\vspace{0.1in}

%\vspace{1cm}

{\bf Problemas:}\\

\begin{questions}
%In order to change the word points to, for example, spanish, punto(s):
\pointpoints{punto}{puntos}

%%%%%%%%%%%%%%%%%%%%%%%%%%%%%%%%%%%%%%%%%%%%%%%%%%%%%%%%%%%%%%%%%%%%%%%%%

\question [1] Clásicamente el momento magnético aparece

\begin{solution}
Este es un problema de simetría esférica...



\end{solution}

vspace{1.0cm}

\question [20]

Un electrón está confinado a permanecer en el interior de una cavidad esférica vacía de radio $R$ con paredes impenetrables. Encuentre una expresión para la presión que dicho electrón en su estado base ejerce sobre las paredes de la cavidad.

\begin{solution}

Este es un problema con simetría esférica bastante simple. La partícula está en el estado base, por lo tanto $\ell =0$. Además, la partícula es libre en el interior.
Podemos hacer uso de la ecuación (6.57) de Zetilli, con $U(r)=rR(r)$, que para $r<R$ se tiene:
\begin{equation*}
-\frac{\hbar^2}{2m}\frac{d^2 U(r)}{dr^2}=E U(r)
\end{equation*} 
donde, $m$ es la masa del electrón. Efectivamente, para el caso de $r \geq R$, la función de onda radial es igual a cero. La ecuación anterior puede escribirse como:
\begin{equation*}
\frac{d^2 U(r)}{dr^2}+k^2U(r)=0
\end{equation*}
donde $k=\frac{\sqrt{2mE}}{\hbar}$. La solución a esta ecuación, como sabemos, es muy simple. Debe ser armónica, es decir, debe tomar la forma $U(r)= A\sin(kr)$, donde $A$ es la constante de normalización.

Esto, por supuesto está de acuerdo con la ecuación 6.64 de Zettili, cuya solución en este caso sería
\begin{equation*}
R(\rho) =Aj_0 (\rho)=A\frac{\sin(\rho)}{\rho}
\end{equation*}
con $\rho=kr$. Para hallar la energía sólo tenemos que satisfacer las condiciones de borde, es decir, $U(r)=0$ para $r=R$. Además, puesto que $R(r)$ tiene que ser finita para todo  $r<R$, entonces $U(0)=0$ (por dicha razón, no funcionaría un coseno).

Por tanto, 
\begin{equation*}
sin(kR)=n\pi  \ \ \ \ \text{con} \quad n=1,2,3, \ldots \notag\quad (\text{n = 0 es trivial})
\end{equation*}
Lo que lleva  a,
\begin{equation*}
k_n=\frac{n\pi}{R}
\end{equation*}

$$U(r)=A\sin(\frac{n\pi r}{R}) $$
Puesto que $k^2=\frac{2mE}{\hbar^2}$, entonces
\begin{equation*}
E_n=\frac{\pi^2\hbar^2n^2}{2mR^2}.
\end{equation*}
Ahora, la presión es la fuerza media por unidad de área. La fuerza es el gradiente del potencial. En este caso, sin embargo toda la energía potencial actuando sobre la pared es la energía cinética del electrón (por conservación de la energía). Entonces
\begin{equation*}
F=-\frac{\partial}{\partial R}\avg{\widehat{H}}=-\frac{\partial E}{\partial R}
\end{equation*}
Al estar el electrón en el estado base, $n=1$
\begin{equation*}
F=- \frac{\partial E_1}{\partial R}=\frac{\pi^2 \hbar^2}{mR^3}
\end{equation*}
Por lo tanto, la presión es

\begin{equation*}
P=\frac{\pi\hbar^2}{4mR^5}
\end{equation*}

\end{solution}



\question [3]

\question [20]
\begin{parts}
    \part[10] Una partícula de espín 1 y otra de espín 2 se encuentran en reposo de manera que su configuración de espín total es 3, con componente $z$ igual a $\hbar$. Si se midiera la componente $z$ del momento angular de la partícula de espín 2, ¿qué valores se podrían obtener y con qué probabilidades? (\textbf{Importante:} en este literal, para construir el o los estados requeridos, no se puede usar las tablas de Clebsch-Gordan ni fórmulas de recurrencia. Se debe mostrar  \textbf{explícitamente} como fueron construidos desde el inicio.)
    
    \part[10] Un electrón con espín ``down'' está en el estado $\psi_{510}$ de un átomo de hidrógeno. Si se pudiera medir el cuadrado del momento angular total del electrón individualmente (sin incluir el espín del protón), ¿qué valores se podría obtener y con qué probabilidades? \textbf{Importante:} aquí sí puede usar las tablas de Clebsch-Gordan o fórmulas de recurrencia. Indique el momento en que las usa.
  \end{parts}
  
  \begin{solution}
    \begin{parts}
      \item Tenemos una caja con $j=3$. Los giróscopos tienen $j_1=3$ y $j_2=1$ (o viceversa).\\*
      Los posibles estados de la caja son:\\*
     \begin{doublespace}
  \(|3,3\rangle ; |3,2\rangle ; |3,1\rangle ; |3,0\rangle ; |3,-1\rangle ; |3,-2\rangle ; |3,-3\rangle \)
  \end{doublespace}
  
  Sin embargo, el problema dice que la proyecci{\' o}n \(z\) de la caja es $\hbar $. Obviamente, ya que { }\(S_z|3,1\rangle = \hbar  |3,1\rangle\),
  eso significa que la caja est{\' a} en el estado \(|3,1\rangle\).
  
  Nos piden obtener probabilidades relacionadas con uno de los gir{\' o}scopos. Entonces, necesitamos expresar el estado de la caja en funci{\' o}n
  de los estados de los gir{\' o}scopos. Como es necesario hacerlo anal{\' \i}ticamente, empiezo por la parte m{\' a}s alta de la escalera, \(|3,3\rangle
  =|2,2\rangle |1,1\rangle\), y aplico el operador \(J_-=J_{1_-}+J_{2_-}\) dos veces para obtener \(|3,1\rangle\).
  
  \[J_-|3,3\rangle =\left(J_{1_-}+J_{2_-}\right)|2,2\rangle |1,1\rangle\]
  
  \begin{align*}
       \sqrt{3(3+1)-3(3-1)}|3,3\rangle &=\sqrt{2(2+1)-2(2-1)}|2,1\rangle |1,1\rangle \\
  &+\sqrt{1(1+1)-1(1-1)}|2,2\rangle |1,0\rangle\
  \end{align*}
  
  \[\sqrt{6}|3,2\rangle =\sqrt{4}|2,1\rangle |1,1\rangle +\sqrt{2}|2,2\rangle |1,0\rangle\]
  
  \[|3,2\rangle =\sqrt{\frac{2}{3}}|2,1\rangle |1,1\rangle +\sqrt{\frac{1}{3}}|2,2\rangle |1,0\rangle\]
  
  Ahora,
  
  \[J_-|3,2\rangle =\left(J_{1_-}+J_{2_-}\right)\left\{\sqrt{\frac{2}{3}}|2,1\rangle |1,1\rangle +\sqrt{\frac{1}{3}}|2,2\rangle |1,0\rangle \right\}\]
  
  \begin{align*}
      \sqrt{3(3+1)-2(2-1)}|3,1\rangle &=\sqrt{\frac{2}{3}}\sqrt{2(2+1)-1(1-1)}|2,0\rangle |1,1\rangle \\ &+\sqrt{\frac{2}{3}}\sqrt{1(1+1)-1(1-1)}|2,1\rangle|1,0\rangle \\
     &+\sqrt{\frac{1}{3}}\sqrt{2(2+1)-2(2-1)}|2,1\rangle |1,0\rangle \\
      &+\sqrt{\frac{1}{3}}\sqrt{1(1+1)-0}|2,2\rangle |1,-1\rangle
  \end{align*}
  
  \begin{align*}
      \sqrt{10}|3,1\rangle &= \sqrt{4}|2,0\rangle |1,1\rangle +\sqrt{\frac{4}{3}}|2,1\rangle |1,0\rangle \\ 
          &+ \sqrt{\frac{4}{3}}|2,1\rangle |1,0\rangle +\sqrt{\frac{2}{3}}|2,2\rangle|1,-1\rangle\\
          &=\sqrt{4}|2,0\rangle |1,1\rangle +2\sqrt{\frac{4}{3}}|2,1\rangle |1,0\rangle +\sqrt{\frac{2}{3}}|2,2\rangle |1,-1\rangle
  \end{align*}
  
  Entonces, los valores que se pueden obtener son:\\
  
  \(2\hbar\) con probabilidad \(\frac{1}{15}\)
  
  \(\hbar\) { }con probabilidad \(\frac{8}{15}\)
  
  \(0\) { }con probabilidad \(\frac{2}{5}\)\\
  
  \item
   Ahora, tenemos una caja que es el electr{\' o}n. Sus dos momentos angulares son los gir{\' o}scopos. Conocemos informaci{\' o}n sobre los gir{\'
  o}scopos individuales (un gir{\' o}scopo tiene esp{\' \i}n {``}hacia abajo{''} y el otro tiene momento angular orbital 1 con proyecci{\' o}n cero)
  y tenemos que obtener probabilidades relacionadas con el momento angular de la caja.
  
  Entonces, a diferencia del literal anterior, es f{\' a}cil obtenerlas observando el estado de los gir{\' o}scopos en funci{\' o}n de los posibles
  estados de la caja. (Es decir, en otra base)
  
  De la tabla de Clebsch-Gordan \(1x \frac{1}{2}\):
  
  \[|1,0\rangle \left|\frac{1}{2},-\frac{1}{2}\right\rangle =\sqrt{\frac{2}{3}}\left|\frac{3}{2},-\frac{1}{2}\right\rangle +\sqrt{\frac{1}{3}}\left|\frac{1}{2},-\frac{1}{2}\right\rangle\]
  
  Ahora, operamos con el cuadrado del momento angular total, que lo calculamos como \(J^2|j,m\rangle =j(j+1)\hbar ^2|j,m\rangle\):
  
  \[J^2|1,0\rangle \left|\frac{1}{2},-\frac{1}{2}\right\rangle =J^2\sqrt{\frac{2}{3}}\left|\frac{3}{2},-\frac{1}{2}\right\rangle +J^2\sqrt{\frac{1}{3}}\left|\frac{1}{2},-\frac{1}{2}\right\rangle\]
  
  \begin{multline*}
      J^2|1,0\rangle \left|\frac{1}{2},-\frac{1}{2}\right\rangle =\left(\frac{3}{2}\left(\frac{3}{2}+1\right)\hbar ^2\right)\sqrt{\frac{2}{3}}\left|\frac{3}{2},-\frac{1}{2}\right\rangle \\
  +\left(\frac{1}{2}\left(\frac{1}{2}+1\right)\hbar ^2\right)\sqrt{\frac{1}{3}}\left|\frac{1}{2},-\frac{1}{2}\right\rangle
  \end{multline*}
  
  \[J^2|1,0\rangle \left|\frac{1}{2},-\frac{1}{2}\right\rangle =\left(\frac{15}{4}\hbar ^2\right)\sqrt{\frac{2}{3}}\left|\frac{3}{2},-\frac{1}{2}\right\rangle
  +\left(\frac{3}{4}\hbar ^2\right)\sqrt{\frac{1}{3}}\left|\frac{1}{2},-\frac{1}{2}\right\rangle\]
  
  Entonces, obtenemos los siguientes valores:\\
  
  \(\frac{15}{4}\hbar\) { }con probabilidad { }\(\frac{2}{3}\)
  
  \(\frac{3}{4}\hbar\) { }con probabilidad { }\(\frac{1}{3}\)
  
  \end{parts}
  \end{solution}
  

\question[20]
Suponga que un electr\'{o}n est\'{a} en un estado descrito por la funci\'{o}n de onda

\begin{equation*}
    \psi=\dfrac{1}{\sqrt{4 \pi}} \left( e^{i \phi} \sin(\theta) + \cos(\theta) \right) g(r),
\end{equation*}
donde  
$$
\int_0 ^{\infty} |g(r)|^2 r^2 dr=1,
$$
y $\phi$, $\theta$ son los \'{a}ngulos azimutal y polar, respectivamente.

\begin{parts}
  \part[10] ?`Cu\'{a}les son los posibles resultados de una medida de la componente $z$, i.e., $L_z$, del momento angular del electr\'{o}n en este estado?
  \part[5] ?`Cu\'{a}l es la probabilidad de obtener cada uno de los posibles valores en el anterior literal?
	\part[5] ?`Cu\'{a}l es el valor esperado de $L_z$?
\end{parts}

\begin{solution}
  \begin{parts}
	\item Como se puede ver en la tabla 5.2 de Zetilli,\\
	
	    $Y_{10}(\phi,\theta)=\sqrt{\dfrac{3}{4 \pi}} cos \theta$\\
	    $Y_{1 \pm 1}(\phi,\theta)=\mp \sqrt{\dfrac{3}{8 \pi}}e^{\pm i \phi} sin \theta$\\
	
	Entonces, la funci\'{o}n de onda puede escribirse como:
 \begin{equation*}
     \psi=\sqrt{\dfrac{1}{3}} \left( -\sqrt{2} Y_{11} + Y_{10}\right)g(r)
 \end{equation*}

 Veamos si $\psi$ esta normalizada:
\small{
\begin{equation*}
\begin{aligned}
 \int \psi^{*} \psi dV&=\dfrac{1}{3}\int \left( -\sqrt{2}Y_{11}^{*} + Y_{10}^{*}\right)g(r)^{*} \left( -\sqrt{2} Y_{11}+Y_{10}\right)g(r) r^2 sin \theta d \theta d\phi dr \\
 &= \dfrac{1}{3}\int_0 ^{\infty} |g(r)|^2 r^2 dr \int \left( 2|Y_{11}|^2 + |Y_{10}|^2  \right)sin \theta d\theta d\phi\\
 &=\dfrac{1}{3} \left( 1\right) \left(2+1\right)\\
 &=1
 \end{aligned}\label{Norm}
\end{equation*}}
\normalsize
Se han utilizado los resultados 5.164 y 5.165 de Zetilli y por lo tanto, los posibles valores de $L_z$ son $\hbar$ y $0$.

\item Como la funci\'{o}n de onda  esta normalizada, la probabilidad de obtener $L_z=\hbar$ es $\left(\sqrt{\dfrac{2}{3}}\right)^2=\dfrac{2}{3}$, mientras que la probabilidad de tener $L_z=0$ es $\left(\dfrac{1}{\sqrt{3}}\right)^2=\dfrac{1}{3}$.

\item 
\begin{equation*}
\begin{aligned}
 \langle L_z \rangle &=\langle \psi| L_z |\psi \rangle \\
 &= \int \psi^{*} L_z \psi r^2 \sin(\theta) d \theta d\phi dr\\
 &=\dfrac{1}{3}\int \left( -\sqrt{2}Y_{11}^{*} + Y_{10}^{*}\right)g(r)^{*} \hat{L_z} \left( -\sqrt{2} Y_{11}+Y_{10}\right)g(r) r^2 sin \theta d \theta d\phi dr\\
\end{aligned}\label{Norm}
\end{equation*}

Puesto que $\hat{L_z}Y_{11}=\hbar Y_{11}$ y que $\hat{L_z}Y_{10}=0$ la integral se transforma en:

\begin{equation*}
    \langle L_z \rangle = \dfrac{2}{3}\hbar \int_{0}^{2\pi} d\phi \int_{0}^{\pi} d\theta sin \theta |Y_{11}|^2
\end{equation*}
	Tomando el resultado 5.165 de Zetilli se tiene $\langle L_z \rangle=\dfrac{2}{3}\hbar$
\end{parts}
\end{solution}








 


\end{questions}

\end{document}
